\documentclass{article}
\usepackage{titlesec}
\usepackage{colortbl}
\usepackage{tabularx}
\usepackage[utf8]{inputenc}

\title{Assignment T2. Peer-To-Peer Network Systems \\
\large TDT4190 - Distributed Systems, spring 2014}

\author{
    Sigve Sebastian Farstad \\
    Christoffer Tønnessen
}

\renewcommand\thesubsection{\alph{subsection})}
\newcommand{\question}[1]{\subsection{}\textit{#1}\bigskip}

\begin{document}

\maketitle

\section{Theory}

\question{In a distributed system with high churn, participating nodes leave and join the system all the times. Is peer-to-peer network or client-server best suited to withstand the high churn? Explain your answer.}

In a peer-to-peer network the role of every node is the same.
When there is a high churn, every node connects itself to others and it doesn't matter who.
This makes for an even distribution of the workload.
In a client-server network all nodes have to connect to the one server.
When a high churn occurs, the workload on the server will increase.
This means the server must spend more time on the nodes connecting and disconnecting.

A peer-to-peer network is therefor better than a client-server network when there is a high churn.

\question{Why are searches in structured peer-to-peer networks limited to lookups on key (which is GUID), while searches in unstructured peer-to-peer network can be anything?}

In structured peer-to-peer networks, the network is organized into a specific topology.
Commonly, this topology is a type of DHT, or Distributed Hash Table.
As hash tables look up values based on a key, it follows naturally that most structured peer-to-peer networks are limited to lookups on keys.

Unstructured peer-to-peer networks, on the other hand, do not implement any efficient distributed global key lookup mechanism.
This means that searches in unstructured peer-to-peer networks need to flood through the network in order to reach as many nodes as possible.
Since key-based lookup provides no advantage in this case, searches might as well be anything.

\section{Practical}

\question{Takes as a starting point a Pastry Network with 12-bit GUID. Suppose there are 25 peers with the following GUIDs: F9C, F89, BFC, BA0, AAE, A7C, 7D9, 7D7, 7D5, 7D4, 4CC, 4CF, 4C1, 4A7, 48E, 39C, 33F, 371, 328, 1EC, 19E, 11E, 170, 129, 0FB. Create the routing tables of the nodes 4CC and 7D9. Each cell in the routing tables will contain either nothing or GUID of a node}

Routing table for 4CC:

\begin{table}[h!]
    \center
    \begin{tabular}{| l | l | l | l | l | l | l |}
    \hline
    - & 0 & -    & 1  & -    & 2   & -    \\
    \hline
    0 & 0 & 0FB  & 40 & -    & 4C0 & -    \\
    \hline
    1 & 1 & 129  & 41 & -    & 4C1 & 4C1  \\
    \hline
    2 & 2 & -    & 42 & -    & 4C2 & -    \\
    \hline
    3 & 3 & 328  & 43 & -    & 4C3 & -    \\
    \hline
    4 & 4 & \cellcolor[gray]{0.8} self & 44 & -    & 4C4 & -    \\
    \hline
    5 & 5 & -    & 45 & -    & 4C5 & -    \\
    \hline
    6 & 6 & -    & 46 & -    & 4C6 & -    \\
    \hline
    7 & 7 & 7D4  & 47 & -    & 4C7 & -    \\
    \hline
    8 & 8 & -    & 48 & 48E  & 4C8 & -    \\
    \hline
    9 & 9 & -    & 49 & -    & 4C9 & -    \\
    \hline
    A & A & A7C  & 4A & 4A7  & 4CA & -    \\
    \hline
    B & B & B80  & 4B & -    & 4CB & -    \\
    \hline
    C & C & -    & 4C & \cellcolor[gray]{0.8} self & 4CC & self \\
    \hline
    D & D & ~    & 4D & -    & 4CD & -    \\
    \hline
    E & E & -    & 4E & -    & 4CE & -    \\
    \hline
    F & F & F89  & 4F & -    & 4CF & 4CF  \\
    \hline
    \end{tabular}
    \caption{Routing table for 4CC}
\end{table}

Tabellen for 7D9:

\begin{table}[h!]
    \center
    \begin{tabular}{| l | l | l | l | l | l | l |}
    \hline
    - & 0 & -    & 1  & -    & 2   & -    \\
    \hline
    0 & 0 & 0FB  & 70 & -    & 7D0 & -    \\
    \hline
    1 & 1 & 129  & 71 & -    & 7D1 & -    \\
    \hline
    2 & 2 & -    & 72 & -    & 7D2 & -    \\
    \hline
    3 & 3 & 328  & 73 & -    & 7D3 & -    \\
    \hline
    4 & 4 & 4CC  & 74 & -    & 7D4 & 7D4  \\
    \hline
    5 & 5 & -    & 75 & -    & 7D5 & 7D5  \\
    \hline
    6 & 6 & -    & 76 & -    & 7D6 & -    \\
    \hline
    7 & 7 & \cellcolor[gray]{0.8} self & 77 & -    & 7D7 & 7D7  \\
    \hline
    8 & 8 & -    & 78 & -    & 7D8 & -    \\
    \hline
    9 & 9 & -    & 79 & -    & 7D9 & \cellcolor[gray]{0.8} self \\
    \hline
    A & A & A7C  & 7A & -    & 7DA & -    \\
    \hline
    B & B & B80  & 7B & -    & 7DB & -    \\
    \hline
    C & C & -    & 7C & -    & 7DC & -    \\
    \hline
    D & D & -    & 7D & self & 7DD & -    \\
    \hline
    E & E & -    & 7E & -    & 7DE & -    \\
    \hline
    F & F & F89  & 7F & -    & 7DF & -    \\
    \hline
    \end{tabular}
    \caption{Rutingtabell for 7D9}
\end{table}


\question{Use the routing table from 2a) to show what happens if the node 7D9 wants to perform a lookup on the value 371. Explain what happens. Assume that each node has a leaf node table of size 2 - that is, each node in addition to the routing table only know about their nearest neighbors with higher GUID and nearest neighbor with lower GUID. Describe any other assumptions you make if any.}

When 371 connects the following would happen, this is assumes the routing table from 2a) has placed 328 in the frist column.

Firstly 7D9 will go through the table and see if 371 is there. It is not and will continue to check all the nodes in the first column. It will chose that node whichs starts on the same prefix as 371. In this table it is 328 and if one would go into 328, 371 would be in that table. 7D9 can connect to 372.

\question{A node with GUID 0FC contacts node 7D9 with a desire to join the Pastry network. What happens?}
7D9 will answer 0FB's request of joining with the first column of it's routing table. 0FC will add the second column and collects it from it's own routing table. 0FB is numerically the asme as 0FC and will afterwards send the message with the routing table and closest leaf node of 0FC back to 0FC. The whole routing table will get sent back to 0FC when the levels are found.

\question{A node has lost connectivity to the network (i.e. the node is "dead" and is no longer available). Describe the most important things that will now happen.}

A node in the Pastry network is considered dead when its immediate neighbors in the node id space can no longer communicate with it.
When this happens, the network needs to eliminate the node from its records.

When a healthy node discovers that one of the nodes in its leaf set is dead, it replaces the dead node with a node from the leaf set of the live node with the largest index on the side of the failed node.

When a healthy node discovers that one of the nodes in its routing table is dead, it routes the message to a different node, and tries to replace the dead node in its routing table.
It does this by asking other similar nodes in its routing table for fitting replacements from their routing tables.

\end{document}
